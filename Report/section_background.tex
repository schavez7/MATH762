\section{Background}
Let $\Omega$ be a region of the $n$th-dimentional real set. The region $\Omega$ will be considered a continuous mass; therefore, a point in $\Omega$ will be considered a mass particle Not to confuse it with an actual physical particle like in quantum physics but rather as just a point with mass. Let $\mathbf{X}$ be the coordinates of point in $\Omega$. These coordinates do not change based on time; they are the coordinates of the point at an initial time which will be given the name \textit{material coordinates}. As the mass particle moves away from those initial coordinates for a certain amount of time $t$, the mapping $\varphi_t(\mathbf{X})$ will give the new coordinates of the mass particle, or \textit{spatial coordinates}, $\mathbf{x}$. As time progresses, the view of the movement is viewed from the reference of the material cooridinates. This is known as the \textit{Lagrangian reference frame}.

After a given time $t$, a displacement field can be produced of all particles in the region
\begin{equation}
  \mathbf{u}(\mathbf{X},t) = \mathbf{x}(\mathbf{X},t) - \mathbf{X}.
\end{equation}
A derivative with respect to the material coordinates gives the \textit{displacement gradient}
\begin{equation}
  \frac{\partial \mathbf{u}}{\partial \mathbf{X}} = \frac{\partial \mathbf{x}}{\partial \mathbf{X}} - \mathbf{I},
\end{equation}
but a simple rearrangement also produces the \textit{deformation gradient tensor}
\begin{equation}
  \mathbf{F} = \frac{\partial \mathbf{x}}{\partial \mathbf{X}} = \frac{\partial \mathbf{u}}{\partial \mathbf{X}} + \mathbf{I}.
\end{equation}
