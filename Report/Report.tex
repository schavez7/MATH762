\documentclass{article}
\usepackage{arxiv}

\usepackage[utf8]{inputenc} % allow utf-8 input
\usepackage[T1]{fontenc}    % use 8-bit T1 fonts
\usepackage{hyperref}       % hyperlinks
\usepackage{url}            % simple URL typesetting
\usepackage{booktabs}       % professional-quality tables
\usepackage{amsfonts}       % blackboard math symbols
\usepackage{nicefrac}       % compact symbols for 1/2, etc.
\usepackage{microtype}      % microtypography
\usepackage{lipsum}

\title{A template for the \emph{arxiv} style}


\author{
  Sergio ~Chavez\thanks{Main author.} \\
  Department of Mathematics\\
  University of North Carolina at Chapel Hill\\
  Chapel Hill, NC 27599 \\
  \texttt{schavez1@email.unc.edu} \\
}

\begin{document}
\maketitle

\begin{abstract}
\lipsum[1]
\end{abstract}


\section{Introduction}
The goal of this project is provide an example of shear wave propagation through hyperelastic materials using finite element methods. The equations are interest look like
\begin{equation}
  \frac{\partial v}{\partial t} + \nabla\cdot P = 0,
\end{equation}
where $v$ is the velocity and $P$ is the first Piola-Kirchhoff stress tensor (PK1).

\section{Background}
Let $\Omega$ be a region of the $n$th-dimentional real set. The region $\Omega$ will be considered a continuous mass; therefore, a point in $\Omega$ will be considered a mass particle Not to confuse it with an actual physical particle like in quantum physics but rather as just a point with mass. Let $\mathbf{X}$ be the coordinates of point in $\Omega$. These coordinates do not change based on time; they are the coordinates of the point at an initial time which will be given the name \textit{material coordinates}. As the mass particle moves away from those initial coordinates for a certain amount of time $t$, the mapping $\varphi_t(\mathbf{X})$ will give the new coordinates of the mass particle, or \textit{spatial coordinates}, $\mathbf{x}$. As time progresses, the view of the movement is viewed from the reference of the material cooridinates. This is known as the \textit{Lagrangian reference frame}.

After a given time $t$, a displacement field can be produced of all particles in the region
\begin{equation}
  \mathbf{u}(\mathbf{X},t) = \mathbf{x}(\mathbf{X},t) - \mathbf{X}.
\end{equation}
A derivative with respect to the material coordinates gives the \textit{displacement gradient}
\begin{equation}
  \frac{\partial \mathbf{u}}{\partial \mathbf{X}} = \frac{\partial \mathbf{x}}{\partial \mathbf{X}} - \mathbf{I},
\end{equation}
but a simple rearrangement also produces the \textit{deformation gradient tensor}
\begin{equation}
  \mathbf{F} = \frac{\partial \mathbf{x}}{\partial \mathbf{X}} = \frac{\partial \mathbf{u}}{\partial \mathbf{X}} + \mathbf{I}.
\end{equation}



\section{Examples of citations, figures, tables, references}
\label{sec:others}
\lipsum[8] \cite{kour2014real,kour2014fast} and see \cite{hadash2018estimate}.

The documentation for \verb+natbib+ may be found at
\begin{center}
  \url{http://mirrors.ctan.org/macros/latex/contrib/natbib/natnotes.pdf}
\end{center}
Of note is the command \verb+\citet+, which produces citations
appropriate for use in inline text.  For example,
\begin{verbatim}
   \citet{hasselmo} investigated\dots
\end{verbatim}
produces
\begin{quote}
  Hasselmo, et al.\ (1995) investigated\dots
\end{quote}

\begin{center}
  \url{https://www.ctan.org/pkg/booktabs}
\end{center}


\subsection{Figures}
\lipsum[10]
See Figure \ref{fig:fig1}. Here is how you add footnotes. \footnote{Sample of the first footnote.}
\lipsum[11]

\begin{figure}
  \centering
  \fbox{\rule[-.5cm]{4cm}{4cm} \rule[-.5cm]{4cm}{0cm}}
  \caption{Sample figure caption.}
  \label{fig:fig1}
\end{figure}

\subsection{Tables}
\lipsum[12]
See awesome Table~\ref{tab:table}.

\begin{table}
 \caption{Sample table title}
  \centering
  \begin{tabular}{lll}
    \toprule
    \multicolumn{2}{c}{Part}                   \\
    \cmidrule(r){1-2}
    Name     & Description     & Size ($\mu$m) \\
    \midrule
    Dendrite & Input terminal  & $\sim$100     \\
    Axon     & Output terminal & $\sim$10      \\
    Soma     & Cell body       & up to $10^6$  \\
    \bottomrule
  \end{tabular}
  \label{tab:table}
\end{table}

\subsection{Lists}
\begin{itemize}
\item Lorem ipsum dolor sit amet
\item consectetur adipiscing elit.
\item Aliquam dignissim blandit est, in dictum tortor gravida eget. In ac rutrum magna.
\end{itemize}


\bibliographystyle{unsrt}
\bibliography{references}


\end{document}
