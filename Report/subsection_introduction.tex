\subsection{Introduction}
The ultimate goal of this project is to produce wave propagation through a hyperelastic material which can be given by the following hyperbolic equation
\begin{equation}
  \label{eq::Elastic_wave}
  \frac{\pd^2 \bm{u}}{\pd t^2} - \nabla\cdot \bsig = \bm{b}.
\end{equation}
The details of this equation will be explained further below.

The example for the first milestone uses an implicit method to solve the time step integration but for the elastic model an explicit method will be used to facilitate a generalisation for any hyperelastic model. A first-stage Runge-Kutta explicit method will be used
\begin{eqnarray*}
  U^* & = & U^n + \Delta tf(U^n,t_n), \\
  U^{n+1} & = & U^n + \Delta tf(U^*,t_n+k/2).
\end{eqnarray*}
Transforming the equation (\ref{eq::Elastic_wave}) into a system of first-order differential equations so to use the Runge-Kutta method gives
\begin{eqnarray*}
  \frac{\pd \bm{u}}{\pd t} & = & \bm{v}, \\
  \frac{\pd \bm{v}}{\pd t} & = & \nabla\cdot\bsig + \bm{f}, \\
  \frac{\pd \bsig}{\pd t} & = & \bm{h}(\bm{v},t),
\end{eqnarray*}
where $\bm{h}$ is the derivative of the hyperelastic material model with respect to time and space. In this way, the stress tensor is left general based on the model chosen.
