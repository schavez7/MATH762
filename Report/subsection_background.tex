\subsection{Background}
Let $\Omega$ be a region of the $n$th-dimensional real set. The region $\Omega$ will be considered a continuous mass; therefore, a point in $\Omega$ will be considered a mass particle. Though, not to confuse this particle with those of quantum physics, but rather as just a point with mass. Let $\bm{X}$ be the coordinates of a point in $\Omega$. These coordinates do not change based on time; they are the coordinates of a point at an initial time which will be given the name \textit{material coordinates}. As the mass particle moves away from its initial position for a certain amount of time $t$, the mapping $\varphi_t(\bm{X})$ will give the new coordinates of the mass particle, or \textit{spatial coordinates}, $\bm{x}$. As time progresses, the view of the movement is viewed from the reference of the material coordinates. This approach is known as the \textit{Lagrangian reference frame}.

After a given time $t$, a displacement field can be produced of all particles in the region
\begin{equation}
  \bm{u}(\bm{X},t) = \bm{x}(\bm{X},t) - \bm{X}.
\end{equation}
A derivative with respect to the material coordinates gives the \textit{displacement gradient}
\begin{equation}
  \frac{\pd\bm{u}}{\pd\bm{X}} = \frac{\pd\bm{x}}{\pd\bm{X}} - \bm{I},
\end{equation}
but a simple rearrangement also produces the \textit{deformation gradient tensor}
\begin{equation}
  \bm{F} = \frac{\pd\bm{x}}{\pd\bm{X}} = \frac{\pd\bm{u}}{\pd\bm{X}} + \bm{I}.
\end{equation}

The stress-strain relationship according to Hooke's law gives
\begin{equation}
  \label{eq::Hookes_Law}
  \bsig = \mathbb{C}:\beps,
\end{equation}
where $\mathbb{C}$ is a fourth-order elastic tensor.
